% ----------------------------------------------------
% Introduction
% ----------------------------------------------------
\documentclass[class=report,11pt,crop=false]{standalone}
% Page geometry
\usepackage[a4paper,margin=20mm,top=25mm,bottom=25mm]{geometry}

% Font choice
\usepackage{lmodern}

% Use IEEE bibliography style
\bibliographystyle{IEEEtran}

% Line spacing
\usepackage{setspace}
\setstretch{1.20}

% Ensure UTF8 encoding
\usepackage[utf8]{inputenc}

% Language standard (not too important)
\usepackage[english]{babel}

% Skip a line in between paragraphs
\usepackage{parskip}

% For the creation of dummy text
\usepackage{blindtext}

% Math
\usepackage{amsmath}

% Header & Footer stuff
\usepackage{fancyhdr}
\pagestyle{fancy}
\fancyhead{}
\fancyhead[R]{\nouppercase{\rightmark}}
\fancyfoot{}
\fancyfoot[C]{\thepage}
\renewcommand{\headrulewidth}{0.0pt}
\renewcommand{\footrulewidth}{0.0pt}
\setlength{\headheight}{13.6pt}

% Epigraphs
\usepackage{epigraph}
\setlength\epigraphrule{0pt}
\setlength{\epigraphwidth}{0.65\textwidth}

% Colour
\usepackage{color}
\usepackage[usenames,dvipsnames]{xcolor}

% Hyperlinks & References
\usepackage{hyperref}
\definecolor{linkColour}{RGB}{77,71,179}
\hypersetup{
    colorlinks=true,
    linkcolor=linkColour,
    filecolor=linkColour,
    urlcolor=linkColour,
    citecolor=linkColour,
}
\urlstyle{same}

% Automatically correct front-side quotes
\usepackage[autostyle=false, style=ukenglish]{csquotes}
\MakeOuterQuote{"}

% Graphics
\usepackage{graphicx}
\graphicspath{{Images/}{../Images/}}
\usepackage{makecell}
\usepackage{transparent}

% SI units
\usepackage{siunitx}

% Microtype goodness
\usepackage{microtype}

% Listings
\usepackage[T1]{fontenc}
\usepackage{listings}
\usepackage[scaled=0.8]{DejaVuSansMono}

% Custom colours for listings
\definecolor{backgroundColour}{RGB}{250,250,250}
\definecolor{commentColour}{RGB}{73, 175, 102}
\definecolor{identifierColour}{RGB}{196, 19, 66}
\definecolor{stringColour}{RGB}{252, 156, 30}
\definecolor{keywordColour}{RGB}{50, 38, 224}
\definecolor{lineNumbersColour}{RGB}{127,127,127}
\lstset{
  language=Matlab,
  captionpos=b,
  aboveskip=15pt,belowskip=10pt,
  backgroundcolor=\color{backgroundColour},
  basicstyle=\ttfamily,%\footnotesize,        % the size of the fonts that are used for the code
  breakatwhitespace=false,         % sets if automatic breaks should only happen at whitespace
  breaklines=true,                 % sets automatic line breaking
  postbreak=\mbox{\textcolor{red}{$\hookrightarrow$}\space},
  commentstyle=\color{commentColour},    % comment style
  identifierstyle=\color{identifierColour},
  stringstyle=\color{stringColour},
   keywordstyle=\color{keywordColour},       % keyword style
  %escapeinside={\%*}{*)},          % if you want to add LaTeX within your code
  extendedchars=true,              % lets you use non-ASCII characters; for 8-bits encodings only, does not work with UTF-8
  frame=single,	                   % adds a frame around the code
  keepspaces=true,                 % keeps spaces in text, useful for keeping indentation of code (possibly needs columns=flexible)
  morekeywords={*,...},            % if you want to add more keywords to the set
  numbers=left,                    % where to put the line-numbers; possible values are (none, left, right)
  numbersep=5pt,                   % how far the line-numbers are from the code
  numberstyle=\tiny\color{lineNumbersColour}, % the style that is used for the line-numbers
  rulecolor=\color{black},         % if not set, the frame-color may be changed on line-breaks within not-black text (e.g. comments (green here))
  showspaces=false,                % show spaces everywhere adding particular underscores; it overrides 'showstringspaces'
  showstringspaces=false,          % underline spaces within strings only
  showtabs=false,                  % show tabs within strings adding particular underscores
  stepnumber=1,                    % the step between two line-numbers. If it's 1, each line will be numbered
  tabsize=2,	                   % sets default tabsize to 2 spaces
  %title=\lstname                   % show the filename of files included with \lstinputlisting; also try caption instead of title
}

% Caption stuff
\usepackage[hypcap=true, justification=centering]{caption}
\usepackage{subcaption}

% Glossary package
% \usepackage[acronym]{glossaries}
\usepackage{glossaries-extra}
\setabbreviationstyle[acronym]{long-short}

% For Proofs & Theorems
\usepackage{amsthm}

% Maths symbols
\usepackage{amssymb}
\usepackage{mathrsfs}
\usepackage{mathtools}

% For algorithms
\usepackage[]{algorithm2e}

% Spacing stuff
\setlength{\abovecaptionskip}{5pt plus 3pt minus 2pt}
\setlength{\belowcaptionskip}{5pt plus 3pt minus 2pt}
\setlength{\textfloatsep}{10pt plus 3pt minus 2pt}
\setlength{\intextsep}{15pt plus 3pt minus 2pt}

% For aligning footnotes at bottom of page, instead of hugging text
\usepackage[bottom]{footmisc}

% Add LoF, Bib, etc. to ToC
\usepackage[nottoc]{tocbibind}

% SI
\usepackage{siunitx}

% For removing some whitespace in Chapter headings etc
\usepackage{etoolbox}
\makeatletter
\patchcmd{\@makechapterhead}{\vspace*{50\p@}}{\vspace*{-10pt}}{}{}%
\patchcmd{\@makeschapterhead}{\vspace*{50\p@}}{\vspace*{-10pt}}{}{}%
\makeatother
\makenoidxglossaries
% --------------------------------------------------------------------
% Examples of creating a glossary
\newacronym{cw}{CW}{Continuous-Wave}
\newacronym{dsp}{DSP}{Digital Signal Processing}
\newacronym{em}{EM}{Electromagnetic}
\newacronym{fmcw}{FMCW}{Frequency Modulated Continuous Wave}
\newacronym{gui}{GUI}{Graphical User Interface}
\newacronym{rf}{RF}{Radio Frequency}
\newacronym{radar}{RADAR}{Radio Detection and Ranging}
\newacronym{pcb}{PCB}{Printed Circuit Board}
\newacronym{pc}{PC}{Personal Computer}
\newacronym{pri}{PRI}{Pulse Repetition Interval}
\newacronym{adc}{ADC}{Analogue-to-Digital Converter}
\newacronym{if}{IF}{Intermediate Frequency}
\newacronym{itu}{ITU}{International Telecommunications Union}
\newacronym{rcs}{RCS}{Radar Cross Section}
\newacronym{opamp}{Op Amp}{Operational Amplifier}
\newacronym{gbwp}{GBWP}{Gain Bandwidth Product}
\newacronym{dc}{DC}{Direct Current}
\newacronym{ac}{AC}{Alternating Current}
\newacronym{uct}{UCT}{University of Cape Town}
\newacronym{usb}{USB}{Universal Serial Bus}
\newacronym{stft}{STFT}{Short-Time Fourier Transform}
\newacronym{fft}{FFT}{Fast Fourier Transform}
\newacronym{dft}{DFT}{Discrete Fourier Transform}
\newacronym{dtft}{DTFT}{Discrete-Time Fourier Transform}
\newacronym{snr}{SNR}{Signal-to-Noise Ratio}
\newacronym{prf}{PRF}{Pulse Repetition Frequency}
\newacronym{isar}{ISAR}{Inverse Synthetic Aperture}
% include SUV (check experimentation table)
% --------------------------------------------------------------------

\begin{document}
% ----------------------------------------------------
\chapter{Introduction \label{ch:introduction}}
\vspace{-1cm}
% ----------------------------------------------------
This report documents the design, development, and implementation of an ultrasonic \gls{radar} demonstrator oriented towards measuring the speed of slow-moving vehicles. In this application area, radar offers solutions to reliably determine how far an object is and provides accurate measurements of the relative speed of the target. Ultrasonic radar technology is well suited for this application area due to its ability to detect low speeds of a target at short distances. The use of ultrasonic \gls{radar} systems for object speed determination is not a novel idea. Commercial systems are limited in this application area, but there are available experimental systems developed and presented through research. Carullo and Parvis present a design for using an ultrasonic sensor to measure the distance to a target vehicle \cite{carullo}. Odat et al. present a vehicle classification, detection, and speed estimation device using a combination of passive infrared and ultrasonic sensors \cite{enas}.

\section{Background}
Edwards provides a design of an ultrasonic radar demonstrator for detecting the speed of moving sports balls\cite{ian}. This project provided a comprehensive design of a working, low-cost demonstrator that can accurately measure the speed of moving sports balls. The bigger picture involves using ultrasonic radar for short-range, low-speed target detection. The work in this project is focused on addressing the use of ultrasonic \gls{radar} sensors to detect moving vehicles and accurately measure their speed and low ranges. One important real-life application of this work is using the radar to detect the speed of vehicles driving over the speed limit in parking areas; drivers can then be issued tickets or warnings, depending on the location of the parking. Although their application is limited in the market, they present an affordable and reliable alternative to the radar detectors available today. The application of this work can be extrapolated to vehicle detection systems, such as the pre-crash system provided by Sun et al. \cite{sun} which is designed using a low-light camera. Experimental vehicle detection systems often make use of vision-based detection which provides workable results under normal conditions, but fails to perform well under inclement weather conditions \cite{sun} \cite{cheon}. Ultrasonic sensors do not have the disadvantage of being affected by weather conditions as they work with electromagnetic waves which are capable of travelling through air, solid objects, and space \cite{pomr}. Vehicle detection systems contribute to many applications such as driver-assistance systems, automatic parking systems, and self-guided vehicles \cite{cheon}. Hence, this work has the potential to be used in applications that reduce driving difficulty and traffic accidents and provides means to improve the driving experience.

\section{Problem Statement}
Ultrasonic \gls{radar} systems are a trustworthy choice for building object-detection systems, however, they come with some challenges of their own. Some challenges involved in building ultrasonic \gls{radar} systems are inclusive of designing a low-cost, well-packaged, easy-to-deploy radar demonstrator. Once a suitable demonstrator is built, there is the added challenge of identifying suitable locations and/or positions to deploy or install it to provide accurate results; even when accurate results can be obtained, signal processing techniques can pose an issue if the demonstrator is designed to measure the speed of moving vehicles. Characterising the functionality of the radar demonstrator in a controlled and uncontrolled environment and identifying the limits at which the demonstrator can accurately measure the speed of moving objects, such as the maximum speed and the maximum range of the moving object, are added problems when looking to design a solution. Hence, determining the range of the moving object requires appropriate signal processing techniques to detect the return signal from the moving object. Additionally, the signal processing must effectively attenuate noise and clutter.

There has been some effective work done to address these problems when designing an effective and accurate ultrasonic radar system. One such example is an ultrasonic radar system designed to detect the speed of moving sports balls \cite{ian}. This design was low-cost, well-packaged, and tested in indoor and outdoor environments (with better results shown in outdoor environments). The design accurately detected the speed of moving sports balls, but much of the accuracy was limited to the size (the \gls{rcs}) of the balls. Due to effective filtering, noise and clutter did not have a large effect on the perceived results, but higher velocity balls showed lower accuracy in measuring the speed. To provide accurate speed measurements, three algorithms were analysed and one was selected. Unfortunately, the narrow bandwidth and low \gls{rcs} of some of the balls, limited the accuracy of the speed measurements and the radar's ability to detect some sports balls. Lin provides a similar approach in his dissertation detailing a dynamic hand gesture recognition system using ultrasonic sensors \cite{clin}.

Although there are examples of approaches that address a number of the issues mentioned above, there is still some insufficient work done to address some problems. For example, Edwards' ultrasonic design was unable to accurately detect sports balls moving at speeds greater than 10 m/s \cite{ian}. This is because most readily available, and low-cost, ultrasonic sensors have a narrow-band between 1kHz to 3 kHz. This significantly attenuates frequencies that fall outside the sensor's critical frequencies. Hence, ultrasonic radars are limited in their ability to detect higher-velocity objects. 

With these issues defined, this project aimed to address some of the problems that arise when designing ultrasonic radar systems for object speed determination. This project aimed to improve on the design provided by Edwards \cite{ian} to develop an ultrasonic radar demonstrator to accurately detect the radial speed of slow-moving cars in parking areas. The functional limits of the radar were tested and analysed, such as the maximum speed that can be measured unambiguously and the maximum size of car that can be measured by the demonstrator. The demonstrator was designed to be able to accurately detect cars moving at speeds of at most 20 km/hr at a distance of at least 10m. The final design was then mounted onto a stripboard. It was assumed that, because vehicles have a higher \gls{rcs} than sports balls, they will be more accurately detected.

This project yields great benefits for multiple applications. By measuring the speed of vehicles, vehicles that travel at greater than the indoor speed limit can be identified and later fined for not complying with the rules. This can be implemented in areas such as malls, apartment blocks, and even indoor and outdoor school parking areas. This implementation can be extended for use in detecting the speeds of larger moving objects or for use in driving-aid and active safety systems.

\section{Objectives}
The objectives of the project, and presented in this report, are as follows:
\begin{enumerate}
    \item Design a working radar demonstrator that includes the transmitter, receiver, and a working power supply.
    \item Accurately determine the speed of slow-moving cars in a parking area up to an accuracy of ± 2 km/h at 10 m.
    \item Accurately detect the speed of moving cars for a speed of at most 20 km/h.
\end{enumerate}

\section{Scope and Limitations}
This project was limited to a time constraint of a roughly 14-week project lifespan which includes its development and testing as well as a R2000.00 budget.  

The scope of the project includes a well-designed ultrasonic radar demonstrator that is well-packaged and performs well within reasonable acceptance. This demonstrator should comprise of a portable 40 kHz CW radar for vehicle speed measurement along with a stripboard. Furthermore, signal processing should be performed in \textsc{Matlab} and include a basic \gls{gui} for easy user interaction and to display the results of the experiments.

For the purpose of evaluating the performance of the demonstrator, vehicles of various sizes - categorised by their dimensions - will be used in the experiments. The environment in which the experiments are conducted in will affect the perform of the demonstrator, hence, to analyse the effect of an environment on the results, tests will be conducted in an indoor and an outdoor parking area.

Pre-recorded data is not a part of this project's scope; instead, the demonstrator is required to be tested in the field \footnote{The \emph{field}, in this case, refers to indoor and outdoor parking areas used to test the performance of the demonstrator.} and its results validated in the report. This project aims to utilise the ultrasonic radar design detailed by Edwards \cite{ian} and extrapolate it for use in vehicle speed-monitoring cases. 

\section{Report Outline}
% or call it 'plan of development'
This report begins with an introduction to the project which details the research problem statement, the objectives, and the scope and limitations to the study in Chapter~\ref{ch:introduction}. A broad look at literature relevant to the study is then provided in Chapter~\ref{ch:literature}. Once this groundwork has been set, the user requirements, technical specifications, and acceptance test procedure are provided in Chapter~\ref{ch:requirements}. Thereafter, the system of principles, techniques, and procedures used in the development of this project is defined in the methodology section in Chapter~\ref{ch:methodology} before providing a thorough presentation of the design of the ultrasonic radar demonstrator, including the relevant block diagrams and system design in Chapter~\ref{ch:design}. Chapter~\ref{ch:testing} provides a detailed analysis of the tetsing conducted on the designs established in Chapter~\ref{ch:design}. Once the designs have been built and tested and their results validated, Chapter\ref{ch:experimentation} explains how the demonstrator was deployed and the experiments were conducted. The results section in Chapter~\ref{ch:results} then outlines the results collected from the experiments outlined in Chapter\ref{ch:experimentation} and provides an in-depth look into the readiness and performance of the demonstrator. Finally, conclusions are made and recommendations are provided in Chapter~\ref{ch:conclusions} and Chapter~\ref{ch:recommendations}, respectively. Appendices are attached to the end of the report with information related to the report.

% ----------------------------------------------------
\ifstandalone
\bibliography{../Bibliography/References.bib}
\printnoidxglossary[type=\acronymtype,nonumberlist]
\fi
\end{document}
% ----------------------------------------------------